\documentclass{article}
\usepackage{0/0_note_style}


% Title page customization
\setLogo{0/title/Logo_Politecnico_Milano.png}
\setCourseTitle{Course Title}
\setYear{Academic Year}
\setAuthorCustom{Author}
\setFooter{University}
\setDegree{Degree}



\begin{document}
    %%%%%%%%%%%%
    % Glossary %
    %%%%%%%%%%%%
    %%%%%%%%%%%%%%%%%%%%
% Glossary Entries %
%%%%%%%%%%%%%%%%%%%%
\newglossaryentry{genericGlossaryEntry}{
    name={generic glossary entry},
    description={generic glossary entry description}
}
\newglossaryentry{nge:newGlossaryEntry}{
    type={newGlossary},
    name={new glossary entry},
    description={new glossary description}
}


%%%%%%%%%%%%
% Acronyms %
%%%%%%%%%%%%%
\newacronym{acr}{ACR}{acronym}




%%%%%%%%%%%%%%%%%%%%%%
% Print All Glossary %
%%%%%%%%%%%%%%%%%%%%%%
\printglossaries


    %%%%%%%%%%%%%%
    % Title Page %
    %%%%%%%%%%%%%%
    \makeCustomTitle
    
    
    %%%%%%%%%%%%
    % Contents %
    %%%%%%%%%%%%
    \tableofcontents
    
    
    %%%%%%%%%%%%%%%%
    % All Sections %
    %%%%%%%%%%%%%%%%
    \section{Section}
        In this part of the document the various sections should be added.

        \begin{definition}[Definition] \label{definition label}

            This is a definition. If you want to see the implementation, it is found in \texttt{0/theorem.tex}.
        \end{definition}

        If you want to hide a \textit{List of ...} because you did not use some of them, you can hide it by going to \texttt{0/list\_of\_theorem}.

        \gls{acr}: this is a glossary item, it is found in \texttt{0/list\_of\_glossary.tex}, if you want to add a new type of glossary you add it in \texttt{0/glossary.tex}.

        \cref{definition label}: this is a reference to a definition. \cref{This is an algorithm}: this is a reference to an algorithm.

        \begin{algorithm}[H]
            \scriptsize
            \caption{This is an algorithm} \label{This is an algorithm}
            \begin{algorithmic}[1]
                \Require
                    \Statex $a$ \Comment{number}
                    \Statex $b$ \Comment{number}
                \Ensure
                    \Statex $a + b$ \Comment{number}
                \Statex
                
                \Function{algorithm}{a, b}         
                    \State \Return $a + b$
                \EndFunction
            \end{algorithmic}
        \end{algorithm}

        \noindent \scalebox{1}{
             \begin{minipage}[c]{\textwidth}
                \begin{algorithm}[H]
                    \scriptsize
                    \caption{Very long algorithm} \label{4-way merge algorithm}
                    \begin{algorithmic}[1]
                        \Require 
                            \Statex \texttt{matrix} \Comment{unsorted $\texttt{matrix} \in {\mathbb{M}at_{k \times k}}$ whose $\frac{k}{2} \times \frac{k}{2}$-subarrays are roughly sorted}
                        \Ensure
                            \Statex \texttt{matrix} \Comment{roughly sorted $\texttt{matrix} \in {\mathbb{M}at_{k \times k}}$}
                        \Statex
        
                        \For{\texttt{row} in \texttt{matrix}}
                            \If{\texttt{row} in upper half of the \texttt{matrix}}
                                \State $\texttt{sortAscending}{\left(\texttt{row}\right)}$
                            \Else
                                \State $\texttt{sortDescending}{\left(\texttt{row}\right)}$
                            \EndIf
                        \EndFor
                        \For{\texttt{column} in \texttt{matrix}}
                            \State $\texttt{sort}{\left(\texttt{column}\right)}$
                        \EndFor
                        \For{\texttt{row} in \texttt{matrix}}
                            \If{\texttt{row} is odd}
                                \State $\texttt{sortAscending}{\left(\texttt{row}\right)}$
                            \Else
                                \State $\texttt{sortDescending}{\left(\texttt{row}\right)}$
                            \EndIf
                        \EndFor
                        \For{\texttt{column} in \texttt{matrix}}
                            \State $\texttt{sort}{\left(\texttt{column}\right)}$
                        \EndFor
                    \end{algorithmic}
                \end{algorithm}
            \end{minipage}
        }

        \begin{algorithm}[H]
            \scriptsize
            \caption{Rotatesort Algorithm} \label{rotatesort algorithm}
            \begin{multicols}{2}
                \begin{algorithmic}[1]
                    \Require 
                        \Statex \texttt{matrix} \Comment{unsorted $\texttt{matrix} \in {\mathbb{M}at_{n \times n}}$}
                    \Ensure
                        \Statex \texttt{matrix} \Comment{sorted in row-major $\texttt{matrix} \in {\mathbb{M}at_{n \times n}}$}
                    \Statex
    
                    \For{\texttt{verticalSlice} in \texttt{matrix}}
                        \State $\texttt{balance}{\left(\texttt{verticalSlice}\right)}$ \Comment{reduces the number of dirty rows to at most $\sqrt{n}$}
                    \EndFor
                    \State $\texttt{unblock}{\left(\texttt{matrix}\right)}$
                    \For{\texttt{horizontalSlice} in \texttt{matrix}}
                        \State $\texttt{balance}{\left(\texttt{horizontalSlice}\right)}$
                    \EndFor
                    \State $\texttt{unblock}{\left(\texttt{matrix}\right)}$
                    \State $\texttt{shear}{\left(\texttt{matrix}\right)}$
                    \State $\texttt{shear}{\left(\texttt{matrix}\right)}$
                    \State $\texttt{shear}{\left(\texttt{matrix}\right)}$
                    \For{\texttt{row} in \texttt{matrix}}
                        \State $\texttt{sort}{\left(\texttt{row}\right)}$
                    \EndFor
                \end{algorithmic}
            \end{multicols}
        \end{algorithm}

        \begin{table}[H]
            \centering
            \bgroup
            \def\arraystretch{1.5}
                \begin{tabular}{|c|c|c|}
                     \hline
                     & Standard Machine & Variant Machine \\
                     \hline
                     \texttt{randomKeyGuess} & $2^{36} - 2^{33}$ & $2^{36} - 2^{10}$ \\
                     \hline
                     \texttt{machineKeys} & $2^{10} \cdot 2^{23} = 2^{33}$ & $2^{10}$ \\
                    \hline
                \end{tabular}
            \egroup
            \caption{Machine Inputs}
            \label{tab:machine inputs}
        \end{table}


    %%%%%%%%%%%%%%%%%%%
    % List of Theorem %
    %%%%%%%%%%%%%%%%%%%
    %%%%%%%%%%%%%%%%%%%
% List of Theorem %
%%%%%%%%%%%%%%%%%%%
\listoftheorems[ignoreall, show={definition}, swapnumber, title=List of Definition]
\addcontentsline{toc}{section}{List of Definition}


\listoftheorems[ignoreall, show={assumption}, swapnumber, title=List of Assumption]
\addcontentsline{toc}{section}{List of Assumption}


\listoftheorems[ignoreall, show={principle}, swapnumber, title=List of Principle]
\addcontentsline{toc}{section}{List of Principle}


\listoftheorems[ignoreall, show={theorem}, swapnumber, title=List of Theorem]
\addcontentsline{toc}{section}{List of Theorem}

\listoftheorems[ignoreall, show={corollary}, swapnumber, title=List of Corollary]
\addcontentsline{toc}{section}{List of Corollary}


\listoftheorems[ignoreall, show={proposition}, swapnumber, title=List of Proposition]
\addcontentsline{toc}{section}{List of Proposition}


\listoftheorems[ignoreall, show={lemma}, swapnumber, title=List of Lemma]
\addcontentsline{toc}{section}{List of Lemma}


\listoftheorems[ignoreall, show={remark}, swapnumber, title=List of Remark]
\addcontentsline{toc}{section}{List of Remark}


\listoftheorems[ignoreall, show={property}, swapnumber, title=List of Property]    \addcontentsline{toc}{section}{List of Property}


\listoftheorems[ignoreall, show={test}, swapnumber, title=List of Test]
\addcontentsline{toc}{section}{List of Test}


\listoffigures
\addcontentsline{toc}{section}{List of Figure}


\listofalgorithms
\addcontentsline{toc}{section}{List of Algorithms}

    
    
    %%%%%%%%%%%%%%%%%%%%%%
    % Print All Glossary %
    %%%%%%%%%%%%%%%%%%%%%%
    \printglossaries
    
\end{document}
