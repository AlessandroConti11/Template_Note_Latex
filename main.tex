\documentclass{article}
\usepackage{0/0_note_style}


% Title page customization
\setLogo{0/title/Logo_Politecnico_Milano.png}
\setCourseTitle{Course Title}
\setYear{Academic Year}
\setAuthorCustom{Author}
\setFooter{University}
\setDegree{Degree}


\begin{document}
    %%%%%%%%%%%%%%
    % Title Page %
    %%%%%%%%%%%%%%
    \makeCustomTitle
    
    
    %%%%%%%%%%%%
    % Contents %
    %%%%%%%%%%%%
    \tableofcontents
    
    
    %%%%%%%%%%%%%%%%
    % All Sections %
    %%%%%%%%%%%%%%%%
    \section{Section}
        In this part of the document the various sections should be added.

        \begin{definition}[Definition] \label{definition label}

            This is a definition. If you want to see the implementation, it is found in \texttt{0/theorem.tex}.
        \end{definition}

        If you want to hide a \textit{List of ...} because you did not use some of them, you can hide it by going to \texttt{0/list\_of\_theorem}.

        \gls{acr}: this is a glossary item, it is found in \texttt{0/list\_of\_glossary.tex}, if you want to add a new type of glossary you add it in \texttt{0/glossary.tex}.

        \cref{definition label}: this is a reference to a definition. \cref{This is an algorithm}: this is a reference to an algorithm.

        \begin{algorithm}
            \scriptsize
            \caption{This is an algorithm} \label{This is an algorithm}
            \begin{algorithmic}[1]
                \Require
                    \Statex $a$ \Comment{number}
                    \Statex $b$ \Comment{number}
                \Ensure
                    \Statex $a + b$ \Comment{number}
                    
                \Function{algorithm}{a, b}         
                    \State \Return $a + b$
                \EndFunction
            \end{algorithmic}
        \end{algorithm}


    %%%%%%%%%%%%%%%%%%%
    % List of Theorem %
    %%%%%%%%%%%%%%%%%%%
    %%%%%%%%%%%%%%%%%%%
% List of Theorem %
%%%%%%%%%%%%%%%%%%%
\listoftheorems[ignoreall, show={definition}, swapnumber, title=List of Definition]
\addcontentsline{toc}{section}{List of Definition}


\listoftheorems[ignoreall, show={assumption}, swapnumber, title=List of Assumption]
\addcontentsline{toc}{section}{List of Assumption}


\listoftheorems[ignoreall, show={principle}, swapnumber, title=List of Principle]
\addcontentsline{toc}{section}{List of Principle}


\listoftheorems[ignoreall, show={theorem}, swapnumber, title=List of Theorem]
\addcontentsline{toc}{section}{List of Theorem}

\listoftheorems[ignoreall, show={corollary}, swapnumber, title=List of Corollary]
\addcontentsline{toc}{section}{List of Corollary}


\listoftheorems[ignoreall, show={proposition}, swapnumber, title=List of Proposition]
\addcontentsline{toc}{section}{List of Proposition}


\listoftheorems[ignoreall, show={lemma}, swapnumber, title=List of Lemma]
\addcontentsline{toc}{section}{List of Lemma}


\listoftheorems[ignoreall, show={remark}, swapnumber, title=List of Remark]
\addcontentsline{toc}{section}{List of Remark}


\listoftheorems[ignoreall, show={property}, swapnumber, title=List of Property]    \addcontentsline{toc}{section}{List of Property}


\listoftheorems[ignoreall, show={test}, swapnumber, title=List of Test]
\addcontentsline{toc}{section}{List of Test}


\listoffigures
\addcontentsline{toc}{section}{List of Figure}


\listofalgorithms
\addcontentsline{toc}{section}{List of Algorithms}

    
    
    %%%%%%%%%%%%
    % Glossary %
    %%%%%%%%%%%%
    %%%%%%%%%%%%%%%%%%%%
% Glossary Entries %
%%%%%%%%%%%%%%%%%%%%
\newglossaryentry{genericGlossaryEntry}{
    name={generic glossary entry},
    description={generic glossary entry description}
}
\newglossaryentry{nge:newGlossaryEntry}{
    type={newGlossary},
    name={new glossary entry},
    description={new glossary description}
}


%%%%%%%%%%%%
% Acronyms %
%%%%%%%%%%%%%
\newacronym{acr}{ACR}{acronym}




%%%%%%%%%%%%%%%%%%%%%%
% Print All Glossary %
%%%%%%%%%%%%%%%%%%%%%%
\printglossaries
    
\end{document}
